
\documentclass[12pt]{amsart}
\usepackage{geometry} % see geometry.pdf on how to lay out the page. There's lots.
\geometry{a4paper} % or letter or a5paper or ... etc
% \geometry{landscape} % rotated page geometry

% See the ``Article customise'' template for come common customisations

\title{MADAI Workbench 1.8.0 Tutorial: \\A Supplement to the ParaView Tutorial}
\author{Cory Quammen}
\date{} % delete this line to display the current date

%%% BEGIN DOCUMENT
\begin{document}

\maketitle
\tableofcontents

\section{Introduction}

This tutorial will walk you through using several of the features of the MADAI Workbench. These features not available in a plain installation of ParaView.

The content of this tutorial assumes that you have gone through the ParaView Tutorial for version 3.98 or higher of ParaView.

\section{Ensemble Surface Slicing Representation}

In the ParaView tutorial, we have seen that several representations are available for displaying data in ParaView (3D Glyphs, Outline, Points, Surface, Surface with Edges, Wireframe). Another representation is suitable for displaying a group of surfaces using a technique called Ensemble Surface Slicing \cite{Oluwafemi2012}.

\section{Binning Filter}

\section{Gaussian Scalar Splatter Filter}

\section{}

\bibliographystyle{plain}
\bibliography{MADAIWorkbenchTutorial}

\end{document}